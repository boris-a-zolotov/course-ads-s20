\subsection{Более практичная оценка снизу}

Рассмотрим пару $(s_i,i)$ из набора поисковых запросов. Рассмотрим такие точки $(s_j, j)$, $j < i$, которые образуют с $(s_i, i)$ пустой прямоугольник, то есть не содержащий других точек из $P$. Упорядочим все такие точки по второй координате и соединим их $y$-монотонной ломаной сверху вниз, смотреть Рисунок~\ref{fig:vertInt}.

\begin{figure}[h] \centering
\definecolor{specblue}{RGB}{100,15,190}
\newcommand{\dts}{\draw[very thick,->,specblue]}
\tikz[scale=1.12]{
	\draw[->,thick] (-1.5,0) -- (5,0) node[below]{$x$};
	\draw[->,thick] (0,-1.5) -- (0,4.5) node[left]{$y$};
	\coordinate (s) at (2.5,3.25);
	\defrect{(3.25,0.6)}{(s)}{ }{ }
	\defrect{(2,1)}{(s)}{ }{ }
	\defrect{(1.6,1.75)}{(s)}{ }{ }
	\defrect{(3.75,2)}{(s)}{ }{ }
	\defrect{(0.75,2.5)}{(s)}{ }{ }
	\draw (s) -- (2.5,-1.25);
	\dts (s) -- (0.75,2.5); \dts (0.75,2.5) -- (3.75,2);
	\dts (3.75,2) -- (1.6,1.75); \dts (1.6,1.75) -- (2,1);
	\dts (2,1) -- (3.25,0.6);
	\draw (s) node[above]{$s_i \in P$};
}
\caption{Подсчёт числа пересечений с вертикальной прямой}
\label{fig:vertInt}
\end{figure}

Обозначим через $J(s_i)$ количество пересечений этой ломаной с вертикальным лучом, идущим из $s_i$ вниз. Понятно, что такое число можно посчитать для любого элемента последовательности запросов.

\begin{theorem} \label{thm:optIndBound}
\begin{equation} \label{eq:optIndBound}
	\opt(P)\ \ge\ |P| + \sum\limits_{s_i} \frac{J(s_i)}{2}
\end{equation}
\end{theorem}

\begin{proof}
На каждом ребре ломаной, пересекающем вертикальный луч, построим как на диагонали прямоугольник, стороны которого параллельны осям координат. Так у каждого пересечения появится свой прямоугольник. Эти прямоугольники не обязательно касаются друг друга, потому что ломаная может иметь несколько звеньев подряд с одной стороны от вертикального луча. Объединим получившиеся наборы прямоугольников, смотреть Рисунок~\ref{fig:manyInd}.

\newcommand{\rectStack}[2]{
\begin{scope}[xshift=#1 cm]
	\filldraw[draw=black,fill=black] (1.25,1.15)
		circle[radius=0.8mm] node[right]{#2};
	\defrect{(0,0)}{(1.75,0.75)}{ }{ }
	\defrect{(0.25,0)}{(1.5,-0.75)}{ }{ }
	\defrect{(1.25,-0.75)}{(0.5,-1.25)}{ }{ }
	\defrect{(0.75,-1.25)}{(1,-1.75)}{ }{ }
\end{scope}}

\begin{figure}[h] \centering
\tikz[scale=1.19]{
	\rectStack{0}{$s_1$} \rectStack{3.3}{$s_2$} \rectStack{6.6}{$s_3$}
}
\caption{Набор попарно независимых прямоугольников
     (на рисунке отмечены не все точки из $P$)}
\label{fig:manyInd}
\end{figure}

Все прямоугольники в объединении, легко видеть, будут попарно независимы. Осталось лишь применить Теорему~\ref{thm:optFirstBound}.
\end{proof}

\subsection{Оценка снизу через число перебежек}

Фиксируем бинарное дерево поиска $T$ и рассмотрим его вершину $q$. Обозначим через $R(q)$ количество чередований между спусками в левое поддерево $q$ и правое поддерево $q$. Спуски в сам узел $q$ и всё, что происходит вне поддерева $q$, при этом игнорируется.

\begin{theorem} \label{thm:optPereBound}
\begin{equation} \label{eq:optPereBound}
	\opt(P)\ \ge\ \sum\limits_{q \in T} R(q).
\end{equation}
\end{theorem}

\begin{proof}
Следует из Теоремы~\ref{thm:optIndBound}.
\end{proof}

Мы получили нижнюю оценку на $\opt(P)$ довольно изощрённым способом. Действительно ли эта оценка так хороша? Оказывается, есть пример последовательности запросов, который обобщается на любую глубину дерева поиска и делает эту нижнюю оценку бессмысленно большой: $\sum R(Q) = O(n \log n)$ в случае идеально сбалансированного дерева. Это так называемая bit-reversal sequence, смотреть Рисунок~\ref{fig:brs}. Оценку $O(n \log n)$ так-то можно доказать для такой последовательности более простыми способами и на деревьях попроще.

\input{figs/brs}

\section{Tango деревья} \noteauthor{Борис Золотов}

Данная структура данных получила такое название, потому что была изобретена
в самолёте по пути из Нью-Йорка в Буэнос-Айрес, где танго является
очень популярным танцем. Этот факт, а также объяснение принципа работы
структуры данных можно найти в~\cite{demaineTangoVideo}.

Рассмотрим сбалансированное двоичное дерево поиска логарифмической глубины. Если «по-глупому» обработать последовательность из нескольких запросов, то на каждый из них придётся тратить $O (\log n)$ времени, чтобы (в худшем случае, когда ключ лежит в дереве достаточно глубоко) спуститься от корня к листу.

Тем временем, для данного дерева у нас есть нижняя оценка времени работы на нескольких запросах~\eqref{eq:optPereBound}. Мы хотим как можно ближе к ней подойти — так, чтобы время поиска нескольких запросов отличалось от нижней границы если не в константу, то хотя бы в $\log \log$ раз. Тогда мы будем знать, что дерево поиска не может работать быстрее, чем~\eqref{eq:optPereBound}, но может работать не медленнее, чем~\eqref{eq:optPereBound} умножить на $\log \log$.

В сбалансированном двоичном дереве у каждого узла $\mathcal N$ укажем «предпочитаемого потомка»~— того, в которого происходил спуск при прошлом поисковом запросе (при котором вообще происходил спуск через~$\mathcal N$). С помощью выбора у каждого узла предпочитаемого потомка мы можем разбить дерево на предпочитаемые пути, смотреть Рисунок~\ref{fig:tangoTree}. Как это сделать — спустимся из корня по предпочитаемым потомкам, получим один путь от корня к листу. Этот путь режет всё дерево на несколько деревьев поменьше~— рекурсивно вызовемся на каждом из них. \begin{figure}[h] \centering
     \newcommand{\dcn}[1]{\filldraw[fill=black,draw=black] #1 circle[radius=0.8mm];}
     \newcommand{\dar}{\draw[very thick,->,specred]}
     \tikz[scale=0.57]{
	\dar (4.5,6.5) -- (6,4.5); \dar (6,4.5) -- (5.75,2.5); \dar (5.75,2.5) -- (5,0);
	\dar (7.75,2.5) -- (8.75,0); \dar (1.25,2.5) -- (0.25,0);
	\dar (3,4.5) -- (3.25,2.5); \dar (3.25,2.5) -- (2.75,0);
%%
	\foreach \x in {0.25,1.5,2.75,4,5,6.25,7.5,8.75} {\dcn{(\x cm, 0)}}
	\foreach \x in {1.25,3.25,5.75,7.75} {\dcn{(\x cm, 2.5)}}
	\foreach \x in {3,6} {\dcn{(\x cm, 4.5)}} \dcn{(4.5,6.5)}
%%
	\draw[thick] (4.5,6.5) -- (3,4.5) -- (1.25,2.5) -- (1.5,0)
	     (6,4.5) -- (7.75,2.5) -- (7.5,0)
	     (5.75,2.5) -- (6.25,0) (3.25,2.5) -- (4,0);
     }
\caption{Tango-дерево представлено в виде объединения путей}
\label{fig:tangoTree}
\end{figure}

Пусть при реализации некоторого поискового запроса мы прошли по непредпочитаемому ребру. Это значит, что произошла {\it перебежка,} о которой говорилось в теореме~\ref{thm:optPereBound}: в прошлый раз мы спустились в одного потомка, а теперь спускаемся в другого. Это значит, что мы можем заплатить временем работы за проход по непредпочитаемому ребру. А вот внутри предпочитаемых путей нам нужно уметь перемещаться как можно быстрее. Если не константа, то хотя бы $\log \log$...

\subsection{Описание работы структуры}

Будем хранить каждый предпочитаемый путь в виде сбалансированного двоичного дерева~— вот хотя бы красно-чёрного. Нам бы {\it хотелось,} чтобы сортировка внутри красно-чёрного дерева была по глубине залегания элемента в оригинальном дереве~— но мы будем сортировать по ключу, потому что это самая интуитивная сортировка и она не требует хранения дополнительных данных.

Делая из пути красно-чёрное дерево, мы также будем следить за сохранением ссылок между путями — чтобы они были корректно отражены как ссылки между деревьями. При этом ссылка может начать исходить из другого места: то, что раньше было «направо от $c$», теперь «налево от $d$», потому что $c$ и $d$ поменялись уровнями, смотреть Рисунок~\ref{fig:tangoFlyaway}. \begin{figure}[h] \centering
\tikz[scale=0.63]{
     \draw (0,0) node[inner sep=0.18cm](d){$d$}
	++(-3,-1.5) node[inner sep=0.18cm](a){$a$}
	++(2,-1.5) node[inner sep=0.18cm](c){$c$}
	++(-1.5,-1) node[inner sep=0.18cm](b){$b$}
	++(3.5,-0.3) node[inner sep=0.18cm](s){ };
     \draw[thick,specred] (d) -- (a) -- (c) -- (b);
     \draw[decoration=snake,decorate] (c) -- ($(c)!0.82!(s)$);
     \draw[->] ($(c)!0.82!(s)$) -- (s);
}\hspace{1.8cm}
\tikz[scale=0.63]{
     \draw (0,0) node[inner sep=0.18cm](c){$c$}
	++(-3,-1.5) node[inner sep=0.18cm](b){$b$}
	++(-1.5,-1.5) node[inner sep=0.18cm](a){$a$}
	++(7.5,1.5) node[inner sep=0.18cm](d){$d$}
	++(-1.3*1.5 cm, -1.3*1.5 cm) node[inner sep=0.18cm](s){ };
     \draw (a) -- (b) -- (c) -- (d);
     \draw[decoration=snake,decorate] (d) -- ($(d)!0.82!(s)$);
     \draw[->] ($(d)!0.82!(s)$) -- (s);
}
     \caption{Ссылки для перехода между предпочитаемыми
	путями \\ сохраняются, когда мы из пути делаем дерево}
     \label{fig:tangoFlyaway}
\end{figure}
