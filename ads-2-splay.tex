\section{Splay tree}
{\it Оригинальная статья: \cite{tarjan1985splay}}

\subsection{Общая структура дерева}

В этом дереве мы каждый раз, когда захотим что-то сделать с вершиной, будем поднимать ее до корня (операция splay). В самом дереве в этот раз мы можем не хранить ничего, кроме корня root. Но часто хочется уметь быстро считать размер дерева, для этого можно хранить отдельную переменную size для всего дерева.

\begin{algorithmic}[0]
	\algrenewcommand\algorithmicprocedure{\textbf{structure}}
	\Procedure {tree}{}
        \State root
        \State size \Comment{optional}
	\EndProcedure
	\Procedure {node}{}
		\State left, right
		\State key
	\EndProcedure
\end{algorithmic}

Выразим сначала операции insert и erase через операцию splay, а потом будем разбираться со splay. Для erase нам понадобится операция splay\_front(node). Эта операция делает splay для наименьшего ключа в поддереве.

\begin{algorithmic}[1]
	\Procedure {insert}{x}
		\State standard\_insert(x)
		\State splay(x)
    \EndProcedure
    
	\Procedure {get}{x}
		\State splay(x)
    \EndProcedure
    
	\Procedure {erase}{x}
		\State splay(x)
        \State splay\_front(root.right)
        \State standard\_erase(x)
	\EndProcedure
\end{algorithmic}

Два вызова функции splay при удалении нужны для того, чтобы правый сын корневой вершины не имел левого сына (потому что он содержит наименьший ключ в своем поддереве) и операция standard\_erase(x) работала за $\BigO(1)$ (потому что она просто возьмет этого правого сына и поставит на место удаленного корня). Еще стоит отметить, что даже при простом доступе к вершине мы вызываем операцию splay, это нужно потому что наше дерево может иметь довольно большую глубину во время работы, а оценка у нас будет только на амортизированную сложность операции splay.

\subsection{Splay}

\begin{figure}
    \centering
    \caption{Zig}\label{Zig}
    \begin{forest}
        [, phantom, for children={fit=band}, s sep'+=60pt
            [p,circle,draw
                [x,circle,draw
                    [A,tria]
                    [B,tria]
                ]
                [C,tria,name=left]
            ]
            [x,circle,draw
                [A,tria,name=right]
                [p,circle,draw
                    [B,tria]
                    [C,tria]
                ]
            ]
        ]
        \draw[-latex,very thick,shorten <=5mm,shorten >=5mm] (left) to (right);
    \end{forest}    
\end{figure}

Итак, нам надо научиться понимать вершину в корень. Это делается при помощи нескольких видов вращений дерева. Все вращения в дальнейшем будем рассматривать с точностью до симметрии. Простейшее вращение называется zig~(см.~рис.~\ref{Zig}). Легко видеть, что это вращение поднимает вершину $x$ на один уровень выше. При помощи одного этого вращения можно поднять вершину в корень, но для амортизационного анализа нам этого не хватит, поэтому мы будем делать сразу двойные вращения.

\begin{figure}
    \centering
    \caption{Zig-zig}\label{ZigZig}
    \begin{forest}
        [, phantom, for children={fit=band}, s sep'+=60pt
            [g,circle,draw
                [p,circle,draw
                    [x,circle,draw
                        [A,tria]
                        [B,tria]
                    ]
                    [C,tria]
                ]
                [D,tria,name=left]
            ]
            [x,circle,draw
                [A,tria,name=right]
                [p,circle,draw
                    [B,tria]
                    [g,circle,draw
                        [C,tria]
                        [D,tria]
                    ]
                ]
            ]
        ]
        \draw[-latex,very thick,shorten <=5mm,shorten >=5mm] (left) to (right);
    \end{forest}    
\end{figure}
\begin{figure}
    \centering
    \caption{Zig-zag}\label{ZigZag}
    \begin{forest}
        [, phantom, for children={fit=band}, s sep'+=60pt
            [g,circle,draw
            [p,circle,draw
                [A,tria]
                [x,circle,draw
                    [B,tria]
                    [C,tria]
                ]
            ]
            [D,tria,name=left]
        ]
        [x,circle,draw
                [p,circle,draw,name=right
                    [A,tria]
                    [B,tria]
                ]
                [g,circle,draw
                    [C,tria]
                    [D,tria]
                ]
            ]
        ]
        \draw[-latex,very thick,shorten <=5mm,shorten >=5mm] (left) to (right);
    \end{forest}    
\end{figure}

Двойные вращения бывают двух видов: zig-zig (рис.~\ref{ZigZig}) и zig-zag (рис.~\ref{ZigZag}). Оба эти вращения реализуются при помощи пары вращений zig, но для того, чтобы выразить zig-zig, надо сначала выполнить zig от вершины $p$, и только потом от $x$. Zig-zag при этом выражается как два вызова zig от $x$. Стоит отметить, что при splay мы не сможем выполнить двойное вращение, если интересующая нас вершина непосредственный сын корня, тогда мы должны сделать zig и не забыть его посчитать при анализе (но он может быть только один).

Для анализа, мы воспользуемся методом потенциалов. Для начала заведем функцию $w: \text{keys} \to \br_{\geq 1}$. На нее тоже будут какие-то условия. Про то, какой она может быть, поймем позже, пока можно считать, что она всегда возвращает $1$, реально менять ее придется только для следствий. Определим функцию <<размера>> поддерева $s(x) = \sum_{v \in \text{subtree of } x} w(v)$ и функцию <<ранга>> $r(x) = \log_2 s(x)$ (логарифм двоичный, это неожиданно важно, но дальше основание писать не будем), а функцией потенциала всего дерева $T$ будет $\Phi(T) = \sum_{x \in T} r(x)$. Для того, чтобы метод потенциалов работал, нужно чтобы $\Phi$ всегда было неотрицательно. При $w \equiv 1$ это очевидно, а вообще это надо запомнить как первое ограничение на $w$. Амортизированная стоимость операции splay $\text{am.cost} = \Delta\Phi + \#\text{rotations}$ (да, это просто определение). Пусть мы выполнили один splay. Теперь $r(x)$ и $s(x)$ будут обозначать значения до вызова операции, а $r^\prime(x)$ и $s^\prime(x)$~--- после. Тогда на самом деле мы хотим доказать следующую теорему:

\begin{theorem}
    $\text{am.cost} \leq 3(r^\prime(x) - r(x)) + \BigO(1)$
\end{theorem}
\begin{proof}
    Надо оценить $\Delta\Phi$ для каждого из вращений.

    {\it Zig:}
    \begin{align*}
        \Delta\Phi & = r^\prime(p) - r(p) + r^\prime(x) - r(x) \\
        & = r^\prime(p) - r(x) && \text{since $r^\prime(x) = r(p)$} \\
        & \leq r^\prime(x) - r(x) && \text{since $p$ is lower than $x$ after zig}
    \end{align*}
    Дополнительно стоит отметить, что $r^\prime(x) \geq r(x)$ поскольку слева написана сумма по большему множеству, поэтому если мы вдруг захотим это умножить на какую-нибудь произвольно взятую константу $3$, ничего не испортится.

    {\it Zig-zig:}
    \begin{align*}
        \Delta\Phi & = r^\prime(g) - r(g) + r^\prime(p) - r(p) + r^\prime(x) - r(x) \\
        & = r^\prime(g) + r^\prime(p) - r(p) - r(x) \\
        & \leq r^\prime(g) + r^\prime(x) - 2 r(x) && \text{due to the tree structure} \\
        & \leq 3 (r^\prime(x) - r(x)) - 2 && \text{since $r^\prime(g) + r(x) \leq 2(r^\prime(x) - 1)$}
    \end{align*}

    Осталось показать почему $r^\prime(g) + r(x) \leq 2(r^\prime(x) - 1)$.

    \begin{align*}
        \frac{r^\prime(g) + r(x)}{2} & = \log s^\prime(g) + \log s(x) \\
        & \leq \log \parens*{ \frac{s^\prime(x) - w(p)}{2} } && \text{Jensen's inequality} \\
        & = \log \parens*{s^\prime (x) - w(p)} - 1 \\
        & \leq r^\prime(x) - 1
    \end{align*}

    {\it Zig-zag:}
    \begin{align*}
        \Delta\Phi & = r^\prime(g) - r(g) + r^\prime(p) - r(p) + r^\prime(x) - r(x) \\
        & = r^\prime(g) + r^\prime(p) - r(p) - r(x) \\
        & \leq r^\prime(g) + r^\prime(p) - 2 r(x) && \text{due to the tree structure} \\
        & \leq 3 (r^\prime(x) - r(x)) - 2 && \text{since $r^\prime(g) + r^\prime(p) \leq 2(r^\prime(x) - 1)$}
    \end{align*}

    Доказательство неравенства $r^\prime(g) + r^\prime(p) \leq 2(r^\prime(x) - 1)$ в точности повторяет доказательство аналогичного неравенства выше.

    Изменения потенциала от каждого двойного вращения мы оценили как $3(r^\prime(x) - r(x)) - 2$. Все наши страдания были на самом деле направлены на то, чтобы получить двойку в конце. Теперь, когда мы просуммируем по всем вращениям при операции splay, мы получим оценку $\Delta\Phi \leq 3(r^\prime(x) - r(x)) - \#\text{rotations} + \BigO(1)$, поскольку все промежуточные $r(x)$ скомпенсируются, zig будет вызван не более одного раза, а в оценке двойных вращений есть слагаемое $-2$, которые просуммируются в количество вращений. Таким образом, $\text{am.cost} = \Delta\Phi + \#\text{rotations} \leq 3(r^\prime(x) - r(x)) + \BigO(1)$, что нам и надо.
\end{proof}

\todo{Следствия я тоже допишу позже.}
