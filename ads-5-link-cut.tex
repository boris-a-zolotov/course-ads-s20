\section{Link-Cut trees}

\subsection{Описание структуры, план действий} \noteauthoress{Анастасия Софронова}

Наша цель~--- поддерживать структуру данных, которая умеет хранить лес корневых бинарных деревьев и производить с ними следующие операции:

\begin{itemize}
	\item $\operatorname{makeTree}(v)$~--- создать дерево из одной вершины $v$.
	\item $\flink(v, w)$~--- подвесить $u$ к $w$ (при этом $u$ является корнем одного из деревьев леса, а у $w$ не более одного ребёнка).
	\item $\fcut(v)$~--- удалить ребро между $v$ и её родителем.
	\item $\operatorname{findRoot}(u)$~--- найти корень дерева вершины $u$.
	\item $\operatorname{findCost}(u)$~--- возвращает ближайшее к корню ребро минимального веса на пути от $u$ до корня.
	\item $\operatorname{addCost}(u, x)$~--- добавить $x$ к весам всех рёбер на пути от $u$ до корня.
\end{itemize}

При этом findCost можно адаптировать, чтобы искать не минимум на пути, а, например, сумму и т.д.

В \cite{sleator1983linkcut} описано, как добиться асимптотики $\BigO(\log{n})$ на операцию в худшем случае. Мы же изучим link-cut trees, работающие за амортизированное $\BigO(\log{n})$ (\cite{tarjan1984linkcut}).

В описании структуры данных и доказательстве времени работы будет две смысловых части.

\begin{enumerate}
    \item Научиться реализовывать структуру для частного случая дерева~--- пути. А именно, нам потребуются следующие операции:
        \begin{itemize}
            \item $\operatorname{makePath}(v)$~--- создать путь из одной вершины.
            \item $\operatorname{findPath}(v)$~--- вернуть путь, в котором лежит вершина $v$.
            \item $\operatorname{findTail}(p)$~--- найти верхний конец пути $p$ ($p$~--- указатель на нижний конец пути). 
            \item $\fjoin(p, v, q)$~--- объединить пути $p$ и $q$ в один через вершину $v$, т.е., верхний конец пути $p$ и нижний конец пути $q$ соединить с $v$.
            \item $\fsplit(v)$~--- операция, обратная операции join: отрезать рёбра, ведущие из $v$ в предка и в потомка в пути.
            \item $\operatorname{findPathCost}(p)$, $\operatorname{addPathCost}(p, x)$.
        \end{itemize}
    \item Выразить операции на лесе через операции на путях. Т.е., разобьём вершины дерева на пути. После этого некоторые рёбра лежат на путях (сплошные рёбра), а некоторые соединяют разные пути (пунктирные рёбра, состоящие из пар $(\ftail(p), \fsuccessor(p))$). Для операций на дереве нам понадобится также дополнительная функция $\fexpose(v)$, которая превращает путь от $v$ до корня дерева в один из путей разбиения (при этом рёбра, идущие из $v$ вниз, не входят в этот путь).
\end{enumerate}

\subsection{Выражение операций на дереве через операции на путях}

Мы начнём с того, что выразим операции на дереве (разбитом на пути) через операции на путях и $\fexpose(v)$.

\begin{algorithmic}[1]
	\Procedure {makeTree}{$u$}
		\State $\operatorname{makePath}(u)$
	\EndProcedure

	\Procedure {findRoot}{$u$}
		\State $\operatorname{findTail}(expose(u))$
	\EndProcedure

	\Procedure {findCost}{$u$}
		\State $\fexpose(u)$
		\State $\operatorname{findPathCost}(u)$
	\EndProcedure

	\Procedure {addCost}{$u$, $x$}
		\State $\fexpose(u)$
		\State $\operatorname{addPathCost}(u, x)$
	\EndProcedure

	\Procedure {link}{$u$, $w$}
		\State $\fjoin(\varnothing, \fexpose(u), \fexpose(w))$
	\EndProcedure

	\Procedure {cut}{$v$}
		\State $\fexpose(v)$
		\State $\fsplit(v)$
	\EndProcedure

\end{algorithmic}

Таким образом, expose помогает нам свести задачу на дереве к задаче на пути. Мы считаем, что функция expose возвращает указатель на путь, получившийся в результате её исполнения. Некоторых пояснений требует функция $\flink$: здесь мы отождествляем вершину и путь, состоящий только из этой вершины.

Итак, теперь нужно научиться делать expose.

\begin{algorithmic}[1]

	\Procedure {expose}{$u$}
		\State $p \coloneqq \varnothing$ \Comment{Здесь будем накапливать наш текущий путь}
		\While{$u \ne \varnothing$}
			\State $w \coloneqq successor(findPath(u))$ \Comment{Запомним следующий сверху путь в дереве}
			\State $(q, r) \coloneqq \fsplit(u)$ \Comment{Отрежем у $u$ сплошное ребро вниз}
			\If{$q \ne \varnothing$} \Comment{$q$~--- часть пути, проходящего через u, ниже u}
				\State $\fsuccessor(q) \coloneqq u$ \Comment{Теперь ребро из $q$ в $u$~--- пунктирное}
			\EndIf
			\State $p \coloneqq \fjoin(p, u, r)$ \Comment{А ребро из $u$ в наш текущий путь~--- сплошное}
			\State $u \coloneqq w$ \Comment{Перейдём к вершине следующего пути}
		\EndWhile
		\State $\fsuccessor(p) \coloneqq \varnothing$
	\EndProcedure

\end{algorithmic}

Операцию, которая происходит в теле while, назовём $\fsplice$.

\begin{theorem}
Пусть выполнено $m$ операций с лесом, из них $n$ операций $\operatorname{makeTree}$ (т.е., в дереве не более $n$ вершин). Тогда верно следующее:

\begin{enumerate}
    \item Мы произвели $\BigO(m)$ операций с путями деревьев.
    \item $\fexpose$ был вызван $\BigO(m)$ раз.
    \item За все вызовы $\fexpose$ было выполнено $\BigO(m \log{n})$ операций $\fsplice$.
\end{enumerate}
\end{theorem}

\begin{proof}

Первые два пункта очевидно следуют из того факта, что во всех операциях на дереве expose вызывается константное количество раз. Докажем оценку на количество $\fsplice$.

Обозначим $\size(v)$ количество вершин в поддереве вершины $v$.

Назовём ребро $(v, w)$ тяжёлым, если $2 \cdot \size(v) > \size(w)$, и лёгким, если это неравенство не выполняется. Таким образом, на пути от любой вершины до корня дерева не более логарифма лёгких рёбер.

Мы будем рассматривать следующие величины:

\begin{itemize}
	\item HS~--- количество тяжёлых сплошных рёбер в текущий момент времени;
	\item HSC~--- сколько раз мы создавали тяжёлые сплошные рёбра к текущему моменту времени.
\end{itemize}

Каждый $\fsplice$ превращает некоторое пунктирное ребро в сплошное. Будем рассматривать отдельно лёгкие и тяжёлые рёбра. Так как на пути от $u$ до корня не более логарифма лёгких рёбер, то и превратить лёгкое пунктирное в лёгкое сплошное мы могли не более логарифма раз.

Тогда $\# \text{splice} \leq m (\log{n} + 1) + \text{HSC}$.

В конце $\text{HS} \leq n - 1$. Значит, почти все создания тяжёлых сплошных рёбер были <<отменены>>, т.е., если мы создавали HSC тяжёлых сплошных рёбер, то по крайней мере $\text{HSC} - n + 1$ раз мы превратили тяжёлое сплошное в тяжёлое пунктирное.

Это могло произойти во время $\fsplice$, тогда одновременно с этим мы превратили лёгкое пунктирное в лёгкое сплошное. Из этого следует, что $\text{HSC} \leq n - 1 + \frac{m}{2}(\log{n} + 1)$

Итак, мы получили нужную оценку на количество $\fsplice$. По модулю одной маленькой детали: операции $\flink$ и $\fcut$ тоже влияют на наш потенциал HSC.

Во время этих операций лёгкое сплошное ребро могло превратиться в тяжёлое сплошное~--- такие тяжёлые рёбра можно просто не учитывать в значении HSC.

Также тяжёлое сплошное ребро могло превратиться в лёгкое сплошное. Это соответствует уменьшению потенциала, которое при этом не <<уравновешивает>> создание этого тяжёлого ребра в какой-то предыдущий момент времени. Однако, так как на любом пути лёгких рёбер $\BigO(\log{n})$, то на каждую из $m$ операций может произойти не более $\BigO(\log{n})$ <<незарегистрированных>> изменений потенциала.

Суммарно это внесёт в HSC (и нашу итоговую оценку) ещё $\BigO(m \log{n})$ операций.

\end{proof}

\subsection{Операции на путях}

Для реализации операций на путях мы будем использовать Splay-дерево. Будем хранить путь в дереве таким образом, чтобы при обходе дерева dfs-ом мы выписывали путь слева направо, заканчивая вершиной $\ftail$ (таким образом, $\operatorname{findTail}$ будет просто возвращать самую правую вершину дерева). Корень дерева соответствует пути. В узле дерева будем хранить также следующие величины:

\begin{itemize}
	\item $\Delta \fcost(x) = \fcost(x) - \fmincost(x)$, где $\fmincost(x)$~--- это минимальная стоимость вершины в поддереве $x$.
	\item $\Delta \min(x) = \fmincost(x) - \fmincost(p(x))$, а если $x$~--- корень дерева, то $\Delta \min(x) = \fmincost(x)$
\end{itemize}

Здесь $p(x)$~--- предок $x$ в Splay-дереве.

\begin{algorithmic}[1]
	\Procedure {makePath}{$u$}
		\State $\operatorname{makeSplayTree}(u)$
	\EndProcedure

	\Procedure {findPath}{$v$}
		\State $\fsplay(v)$
		\State $\freturn(v)$
	\EndProcedure

	\Procedure {findPathCost}{$v$}
		\While{$\fright(v) \ne 0$ \textbf{and} $\min(\fright(v)) = 0$ \textbf{or} $\fleft(v) \ne 0$ \textbf{and} $\min(\fleft(v)) = 0$}
			\If{$right(v) \ne 0$ \textbf{and} $\min(fright(v)) = 0$}
				\State $v \coloneqq \fright(v)$
			\Else
				\State $v \coloneqq \fleft(v)$
			\EndIf
		\EndWhile
		\State $\fsplay(v)$
		\State $\freturn(v, \Delta \min(v))$
	\EndProcedure

	\Procedure {addPathCost}{$v$, $x$}
		\State $\Delta \min(v) = \Delta \min(v) + x$
	\EndProcedure

	\Procedure {join}{$p$, $v$, $q$}
		\State $v.left = p$
		\State $v.right = q$
	\EndProcedure

	\Procedure {split}{$v$}
		\State $\fsplay(v)$
		\State $\fcut(v, v.left)$
		\State $\fcut(v, v.right)$
	\EndProcedure

\end{algorithmic}

Для анализа мы воспользуемся уже доказанной асимптотикой splay-дерева. Мы рассмотрим <<виртуальное>> splay-дерево, которое будет состоять из всех splay-деревьев путей, а также проведённых между путями пунктирными рёбрами.

Потенциалы будут такими же:

\[iw(v) = \begin{cases}
		 \size(v), & \text{если у $v$ все рёбра в детей~--- пунктирные}\\
		 \size(v) - \size(u), & \text{если $(u, v)$~--- сплошное ребро}
		 \end{cases}\]

\[tw(v) = \sum_{\text{$u$~--- из поддерева $v$ в виртуальном дереве}} iw(u)\]

\[r(v) = \log{tw(v)}\]

\[\Phi = \sum_v r(v)\]

Тогда за одну операцию splay на одном splay-дереве мы платим $3(r(u) - r(v)) + 1$, что даёт амортизированный логарифм, как в анализе асимптотики splay-дерева. Но нам нужно сказать, что на все операции splay во время выполнения одного expose мы суммарно заплатим не более логарифма. Легко видеть, что операция splay не меняет структуры виртуального дерева, а значит, не меняет потенциалы. Таким образом, во время переходов от одного пути к другому во время операции expose все слагаемые $r(v)$, кроме двух, взаимно уничтожатся.
Тогда:

\[\fexpose(v) = 3(r(root) - r(v)) + 2 \# \text{splice},\]

что есть $\BigO(m \log{n})$.
